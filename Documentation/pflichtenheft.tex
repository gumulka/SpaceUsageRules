% Was wir tun wollen, um den Kunden zufrieden zu stellen.
\section{Ziel des Projekts}
\subsection{Erläuterung des zu lösenden Problems}
Die Software wird zur Teilnahme am Wettbewerb ``Informaticup 2015'', der von der Gesellschaft für Informatik ausgreichtet wird, hergestellt.\\
Die Hauptaufgabe der Software ist es, für eine Position, an der sich ein Informationsschild befindet, einen Geltungsbereich zu berechnen.
Hierzu können die öffentlich zugänglichen Daten aus Open Street Map benutzt werden.
\subsection{Wünsche und Prioritäten des Kunden}
Die Veranstalter des Wettbewerbs legen Wert auf eine gute Nachvollziehbarkeit und Dokumentation der Software.
Das abgegebene Programm muss funktionieren, d.h. nicht abstürzen. Es muss aber nicht zwingend eine Erfolgsquote von hundert Prozent einfahren.
Die Dokumentation der Software und des Entstehungsprozesses sind von höherer Priorität.
\subsection{Domänenbeschreibung}
Da es sich bei diesem Projekt um eine Teilnahme an einem Wettbewerb handelt, kann man keinen Einsatzbereich für die Software angeben.
Insofern kommt das Programm hauptsächlich bei der Bewertung durch die Jury zum Einsatz.\\
Ein denkbares Szenario für den Einsatz ist das Kartieren von Flächennutzungsregeln. So muss zum Beispiel ein Nutzer nicht per Hand
den Geltungsbereich eines Rauchverbots in die Karte eintragen. Das Programm errechnet automatisch aus der einer Position einen Bereich, den
der Nutzer bestätigen oder verbessern kann.
\subsection{Maßnahmen zur Anforderungsanalyse}
Unser erster Schritt, um uns mit dem Problem vertraut zu machen, war es den bereitgestellten Testdatensatz zu betrachten.
(Genaueres siehe unten.)
\section{Rahmenbedingungen und Umfeld}
\subsection{Einschränkungen und Vorgaben}
Bei Eingabedaten ist das selbstdefinierte Format der GI zu übernehmen. Bei Ausgabedaten ist das KML-Format vorgeschrieben.
Nutzung von Open Street Map als Kartenmaterial und Java als primäre Programmiersprache.
\subsection{Anwender}
Die Jury ist nach derzeitigem Stand die einzige Anwendergruppe. Darüber hinaus könnte das Programm auf mobilen Geräten Anwendung finden.
\subsection{Schnittstellen und angrenzende Systeme}
Die Software baut auf dem Kartendienst Open Street Map auf.\\
Die Kriterien zur Berechnung des Geltungsbereichs werden in einer Datei gespeichert. Diese ließe sich auch von anderer Software nutzen.
Zum Beispiel, um die darin beschriebenen Regeln automatisch zu erzeugen.
\section{Funktionale Anforderungen}
%\subsection{Use Case-Diagramm}
%TODO Diagramm, wie die verschiedenen Use Cases, Nutzer und Systeme zusammen arbeiten.\\
\subsection{Use Case-Beschreibungen}
% Diese Nummerierung, weil es im Prinzip immer der gleich Anwendungsfall ist, mit leichten Abänderungen.
1: Einfaches Benutzen der Software in der Komandozeile, d.h. mit den Parametern --data, --rules und --outputDir.\\
1.1: Zusätzlich mit dem Parameter --threads\\
1.2: Nur mit dem Parameter --path\\
2: Einfaches Benutzen mit Fehlervisualisierung.
2.1: Zusätzlich mit dem Parameter --overlap
3: Nur mit dem Parameter --help\\


\begin{tabular}{| l | l |}
 \hline
 \textbf{Use Case Nr. 1} & Standardausführung\\
 \hline
 Umfeld & Komandozeile\\
 \hline
 Systemgrenzen & Komandozeile\\
 \hline
 Ebene & komplette Programmbenutzung\\
 \hline
 Hauptakteur & Benutzer\\
 \hline
 Stakeholder u. Interesse & Benutzer will einen Datensatz berechnen lassen.\\
 \hline
 Voraussetzung & Internetverbindung zu OSM. Angegebene Verzeichnisse existieren.\\
	      & Data.txt und Rules.xml befinden sich an angegebener Stelle. \\
 \hline
 Garantien & Sind OSM-Daten unvollständig, wird ein kreisförmiger Geltungsbereich angegeben.\\
	  & Sind SURs unbekannt, gibt es dennoch eine Ausgabe.\\
 \hline
 Erfolgsfall & Es wird eine kml-Datei mit dem berechneten Lösungspolygon im\\
	    & angegebenen Verzeichnis erstellt.\\
 \hline
 Auslöser & Der Befehl wird abgeschickt.\\
 \hline
 Beschreibung & Ein normaler Programmablauf. Folgender Befehl wird abgeschickt.\\
	    & \verb|java -jar InMa.jar --data <file> --rules <file>|\\
	    & \hspace{24pt} \verb|--ouputDir <path>|\\
	    & Daten werden von OSM herunter geladen.\\
	    & Regeln werden angewendet.\\
	    & Lösungen werden im KML-Format gespeichert.\\
 \hline
 Erweiterungen & Mit zusätzlichen oder anderen Parametern. Siehe Use Cases 1.1 bis 1.6\\
 \hline
 Technologie & -\\
 \hline
\end{tabular}



\begin{tabular}{| l | l |}
 \hline
 \textbf{Use Case Nr. 1.1} & Multithreading\\
 \hline
 Umfeld & Komandozeile\\
 \hline
 Systemgrenzen & Komandozeile\\
 \hline
 Ebene & komplette Programmbenutzung\\
 \hline
 Hauptakteur & Benutzer\\
 \hline
 Stakeholder u. Interesse & Benutzer will einen Datensatz berechnen lassen und Zeit sparen,\\
			  & indem die Arbeit auf mehrere Threads verteilt wird.\\
 \hline
 Voraussetzung & Internetverbindung zu OSM. Angegebene Verzeichnisse existieren.\\
	      & Data.txt und Rules.xml befinden sich an angegebener Stelle. \\
 \hline
 Garantien & Sind OSM-Daten unvollständig, wird ein kreisförmiger Geltungsbereich angegeben.\\
	  & Sind SURs unbekannt, gibt es dennoch eine Ausgabe.\\
 \hline
 Erfolgsfall & Es wird eine kml-Datei mit dem berechneten Lösungspolygon im\\
	    & angegebenen Verzeichnis erstellt. Alles in kürzerer Zeit.\\
 \hline
 Auslöser & Abschicken des Befehls\\
 \hline
 Beschreibung & \verb|java -jar InMa.jar --data <file> --rules <file>|\\
	      & \hspace{24pt}\verb|--ouputDir <path> --threads <int>|\\
	      & Threads werden erstellt und arbeiten.\\
 \hline
 Erweiterungen & -\\
 \hline
 Technologie & -\\
 \hline
\end{tabular}


\begin{tabular}{| l | l |}
 \hline
 \textbf{Use Case Nr. 1.2} & verkürzte Standardausführung\\
 \hline
 Umfeld & Komandozeile\\
 \hline
 Systemgrenzen & Komandozeile\\
 \hline
 Ebene & komplette Programmbenutzung\\
 \hline
 Hauptakteur & Benutzer\\
 \hline
 Stakeholder u. Interesse & Benutzer will einen Datensatz berechnen lassen.\\
 \hline
 Voraussetzung & Internetverbindung zu OSM. Angegebene Verzeichnisse existieren.\\
        & Data.txt und Rules.xml befinden sich an angegebener Stelle. \\
 \hline
 Garantien & Sind OSM-Daten unvollständig, wird ein kreisförmiger Geltungsbereich angegeben.\\
    & Sind SURs unbekannt, gibt es dennoch eine Ausgabe.\\
 \hline
 Erfolgsfall & Es wird eine kml-Datei mit dem berechneten Lösungspolygon im\\
      & angegebenen Verzeichnis erstellt.\\
 \hline
 Auslöser & Der Befehl wird abgeschickt.\\
 \hline
 Beschreibung & Folgender Befehl wird abgeschickt.\\
      & \verb|java -jar InMa.jar --path <path>|\\
      & Daten werden von OSM herunter geladen.\\
      & Regeln werden angewendet.\\
      & Lösungen werden im KML-Format gespeichert.\\
 \hline
 Erweiterungen & -\\
 \hline
 Technologie & -\\
 \hline
\end{tabular}



\begin{tabular}{| l | l |}
 \hline
 \textbf{Use Case Nr. 2} & Regelüberprüfung \\
 \hline
 Umfeld & Komandozeile\\
 \hline
 Systemgrenzen & Komandozeile\\
 \hline
 Ebene & komplette Programmbenutzung\\
 \hline
 Hauptakteur & Benutzer\\
 \hline
 Stakeholder u. Interesse & Benutzer will die bisherigen Regeln anhand von Grundwahrheiten überprüfen.\\
 \hline
 Voraussetzung & Internetverbindung zu OSM. Angegebene Verzeichnisse existieren.\\
	      & Data.txt und Rules.xml befinden sich an angegebener Stelle. \\
	      & Es befinden sich $ID$.truth.kml Dateien im gleichen Verzeichnig wie Data.txt \\
 \hline
 Garantien & Sind OSM-Daten unvollständig, wird ein kreisförmiger Geltungsbereich angegeben.\\
	  & Sind SURs unbekannt, gibt es dennoch eine Ausgabe.\\
 \hline
 Erfolgsfall & Es wird eine kml-Datei mit dem berechneten Lösungspolygon im\\
	    & angegebenen Verzeichnis erstellt.\\
	    & Für jedes Polygon, keine Überlappung von mindestens 95\% mit dem \\
	    & Lösungspolygon aufweist werden 2 Bilder erstellt, welche den Fehler visualisieren.\\
 \hline
 Auslöser & Der Befehl wird abgeschickt.\\
 \hline
 Beschreibung & Folgender Befehl wird abgeschickt.\\
	    & \verb|java -jar InMa.jar --data <file> --rules <file>|\\
	    & \hspace{24pt} \verb|--ouputDir <path>|\\
	    & Daten werden von OSM herunter geladen.\\
	    & Regeln werden angewendet.\\
	    & Bilder zu den unzureichenden Lösungen werden gespeichert.\\
	    & Lösungen werden im KML-Format gespeichert.\\
 \hline
 Erweiterungen & siehe 2.1\\
 \hline
 Technologie & -\\
 \hline
\end{tabular}


\begin{tabular}{| l | l |}
 \hline
 \textbf{Use Case Nr. 2.1} & Regelüberprüfung mit varierender Genauigkeit.\\
 \hline
 Umfeld & Komandozeile\\
 \hline
 Systemgrenzen & Komandozeile\\
 \hline
 Ebene & komplette Programmbenutzung\\
 \hline
 Hauptakteur & Benutzer\\
 \hline
 Stakeholder u. Interesse & Benutzer will die bisherigen Regeln anhand von Grundwahrheiten überprüfen.\\
 \hline
 Voraussetzung & Internetverbindung zu OSM. Angegebene Verzeichnisse existieren.\\
        & Data.txt, Rules.xml und overlap.txt befinden sich an angegebener Stelle. \\
        & Es befinden sich $ID$.truth.kml Dateien im gleichen Verzeichnig wie Data.txt \\
 \hline
 Garantien & Sind OSM-Daten unvollständig, wird ein kreisförmiger Geltungsbereich angegeben.\\
    & Sind SURs unbekannt, gibt es dennoch eine Ausgabe.\\
 \hline
 Erfolgsfall & Es wird eine kml-Datei mit dem berechneten Lösungspolygon im\\
      & angegebenen Verzeichnis erstellt.\\
      & Für jedes Polygon, keine Überlappung von mindestens 95\% oder in overlap.txt \\
      & angegebenem Wert mit dem Lösungspolygon aufweist werden 2 Bilder erstellt,\\
      & welche den Fehler visualisieren.\\
 \hline
 Auslöser & Der Befehl wird abgeschickt.\\
 \hline
 Beschreibung & Folgender Befehl wird abgeschickt.\\
      & \verb|java -jar InMa.jar --data <file> --rules <file>|\\
      & \hspace{24pt} \verb|--ouputDir <path>|\\
      & Daten werden von OSM herunter geladen.\\
      & Regeln werden angewendet.\\
      & Bilder zu den unzureichenden Lösungen werden gespeichert.\\
      & Lösungen werden im KML-Format gespeichert.\\
 \hline
 Erweiterungen & -\\
 \hline
 Technologie & -\\
 \hline
\end{tabular}


\begin{tabular}{| l | l |}
 \hline
 \textbf{Use Case Nr. 3} & Hilfetext\\
 \hline
 Umfeld & Kommandozeile\\
 \hline
 Systemgrenzen & Kommandozeile\\
 \hline
 Ebene & komplette Programmausführung\\
 \hline
 Hauptakteur & Benutzer\\
 \hline
 Stakeholder u. Interesse & Der Benutzer möchte sich den Hilfetext ausgeben lassen.\\
 \hline
 Voraussetzung & -\\
 \hline
 Garantien & Der Hilfetext wird angezeigt.\\
 \hline
 Erfolgsfall & Der Hilfetext wird angezeigt.\\
 \hline
 Auslöser & Abschicken des Befehls\\
 \hline
 Beschreibung & Folgender Befehl wird abgeschickt.\\
	      & \verb|java -jar InMa.jar --help|\\
	      & Der Hilfetext wird angezeigt.\\
 \hline
 Erweiterungen & -\\
 \hline
 Technologie & -\\
 \hline
\end{tabular}


% Beispieltabelle:\\
% \begin{tabular}{| l | l |}
%  \hline
%  \textbf{Use Case Nr. 1} & Name des Use Cases\\
%  \hline
%  Umfeld & \\
%  \hline
%  Systemgrenzen & \\
%  \hline
%  Ebene & \\
%  \hline
%  Hauptakteur & \\
%  \hline
%  Stakeholder u. Interesse & \\
%  \hline
%  Voraussetzung & \\
%  \hline
%  Garantien & \\
%  \hline
%  Erfolgsfall & \\
%  \hline
%  Auslöser & \\
%  \hline
%  Beschreibung & \\
%  \hline
%  Erweiterungen & \\
%  \hline
%  Technologie & \\
%  \hline
% \end{tabular}
 %use case diagramm und beschreibung der use cases
\subsection{Technische Anforderungen}
Besondere technische Anforderungen gibt es nicht.\\


\section{Qualitätsanforderungen}
\subsection{Qualitätsziele des Projekts}
\subsubsection{Funktionalität}
Bedeutet hier, dass die funktionalen Anforderungen entsprechend der Aufgabenstellung vollständig und korrekt implementiert werden.

\subsubsection{Zuverlässigkeit}
Es wird garantiert, dass eine fehlende Internetverbindung, ein nicht vorhanden Sein der Eingabedateien oder fehlerhafte Eingabedateien
nicht zu unbestimmtem Verhalten führen.

\subsubsection{Benutzbarkeit}
Das Programm soll für unerfahrene Benutzer durch wenige Komandozeilenparamenter bedienbar sein. Für erfahrene Benutzer soll es
verschiedene Einstellmöglichkeiten bieten.\\
Durch selbsterklärende Ausgaben und Hilfetexte soll das Programm leicht benutzbar sein.

\subsubsection{Effizienz}
Daten, welche von Open Street Map herunter geladen werden, können lokal zwischengespeichert werden, um so die Belastung der Server und
das Datenvolumen zu reduzieren.\\
Die Laufzeit kann durch mehrere Threads reduziert werden.

\subsubsection{Wartbarkeit}
Designentscheidungen sollen dokumentiert, APIs zur Verfügung gestellt werden.

\subsubsection{Portabilität}
Das Programm sollte auf verschiedenen Plattformen laufen, aber es ist nicht erforderlich, dass dies außergewöhnliche Plattformen einschließt.

\subsubsection{Robustheit}
Neue oder unerwartete Space Usage Rules führen nicht zum Absturz, da es einen Fallbackalgorithmus gibt. Wenn Kombinationen von
SURs nicht explizit behandelt werden, so wird die größtmögliche Teilmenge abgedeckt.

\subsection{Qualitäts-Prioritäten des Kunden}
Es wird Wert auf eine umfangreiche Dokumentation gelegt.
\subsection{Wie Qualitätsziele erreicht werden sollen}
Wir benutzen JUnit, um die Funktionalität zu gewährleisten.\\
Um die Qualität der Dokumentation zu sichern, orientieren wir uns an anderen gut dokumentierten Softwareprojekten.
\section{Probleme und Risiken}
Bei Zeitproblemen werden fehlerhafte Eingaben nicht mehr abgefangen. Oder auf die Teilnahme am Wettbewerb wird verzichtet.
\section{Optionen zur Aufwandsreduktion}
\subsection{Inkrementelle Arbeit}
Mögliche Erweiterungen werden nur bei genügend Zeit implementiert:
\begin{itemize}
 \item visualisierung der Metadaten
 \item automatisierte Regelerstellung aus truth-Polygonen
 \item weitere Kartendienste außer Open Street Map
\end{itemize}
