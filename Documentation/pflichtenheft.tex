Was wir tun wollen, um den Kunden zufrieden zu stellen.
\section{Ziel des Projekts}
\subsection{Erläuterung des zu lösenden Problems}
siehe abstract
\subsection{Wünsche und Prioritäten des Kunden}
soll toll sein
\subsection{Domänenbeschreibung}
unbekannt
\subsection{Maßnahmen zur Anforderungsanalyse}
Analyse des Testdatensatzes (siehe unten)
\section{Rahmenbedingungen und Umfeld}
\subsection{Einschränkungen und Vorgaben}
Bei Eingabedaten ist das selbstdefinierte Format der GI zu übernehmen. Bei Ausgabedaten ist das KML-Format vorgeschrieben. 
Nutzung von Open Street Map als Kartenmaterial und Java als primäre Programmiersprache.
\subsection{Anwender}
unbekannt
\subsection{Schnittstellen und angrenzende Systeme}
Zu Open Street Map.
\section{Funktionale Anforderungen}
\subsection{Use Case-Diagramm}
TODO
\subsection{Use Case-Beschreibungen}
TODO
\subsection{Technische Anforderungen}
TODO
\section{Qualitätsanforderungen}
\subsection{Qualitätsziele des Projekts}
\subsubsection{Funktionalität}
Bedeutet hier, dass die funktionalen Anforderungen entsprechend der Aufgabenstellung vollständig und korrekt implementiert werden.

\subsubsection{Zuverlässigkeit}
Es wird garantiert, dass eine fehlende Internetverbindung, ein nicht vorhanden Sein der Eingabedateien oder fehlerhafte Eingabedateien
nicht zu unbestimmtem Verhalten führen.

\subsubsection{Benutzbarkeit}
Das Programm soll für unerfahrene Benutzer durch wenige Komandozeilenparamenter bedienbar sein. Für erfahrene Benutzer soll es
verschiedene Einstellmöglichkeiten bieten.\\
Durch selbsterklärende Ausgaben und Hilfetexte soll das Programm leicht benutzbar sein.

\subsubsection{Effizienz}
Daten, welche von Open Street Map herunter geladen werden, können lokal zwischengespeichert werden, um so die Belastung der Server und 
das Datenvolumen zu reduzieren.\\
Die Laufzeit kann durch mehrere Threads reduziert werden.

\subsubsection{Wartbarkeit}
Designentscheidungen sollen dokumentiert, APIs zur Verfügung gestellt werden. 

\subsubsection{Portabilität}
Das Programm sollte auf verschiedenen Plattformen laufen, aber es ist nicht erforderlich, dass dies außergewöhnliche Plattformen einschließt.

\subsubsection{Robustheit}
Neue oder unerwartete Space Usage Rules führen nicht zum Absturz, da es einen Fallbackalgorithmus gibt. Wenn Kombinationen von 
SURs nicht explizit behandelt werden, so wird die größtmögliche Teilmenge abgedeckt.

\subsection{Qualitäts-Prioritäten des Kunden}
Es wird Wert auf eine umfangreiche Dokumentation gelegt.
\subsection{Wie Qualitätsziele erreicht werden sollen}
Wir benutzen JUnit. TODO
\section{Probleme und Risiken}
Bei Zeitproblemen werden fehlerhafte Eingaben nicht mehr abgefangen. Oder ganz sein gelassen.
\section{Optionen zur Aufwandsreduktion}
\subsection{Inkrementelle Arbeit}
Mögliche Erweiterungen werden nur bei genügend Zeit implementiert:
\begin{itemize}
 \item weitere Kartendienste außer Open Street Map
 \item automatisierte Regelerstellung aus truth-Polygonen
 \item visualisierung der Metadaten
\end{itemize}

\section{Tests}
TODO