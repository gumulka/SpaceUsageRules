\section{Softwareimplementierung}
\subsection{Entwurf}


\subsection{Eingabeformat}
Regeln werden aus einer Textdatei eingelesen, in welcher zeilenweise die Regeln zu je einem Verbot stehen.
Die Datei besteht aus zwei Zeilen, in der ersten steht der Name des Verbotes in der Zweiten die Menge an Regeln.
Die Menge wird durch eckige Klammern verdeutlicht und die einzelnen Regeln durch Kommata getrennt die Trennung von Key und Value erfolgt durch einen Bindestrich.
Soll eine Regel generell für alle Values eines Keys gelten, so kann die Value weggelassen werden und nur der Key angegeben.

Eine Gültige Eingabe kann also folgendermaßen sein:
\begin{lstlisting}[frame=single]
smoking="no" -> [addr:housenumber -> 0.5, landuse - forest -> 1.5]
fishing="no" -> []
\end{lstlisting}
Dieses Regelwerk sagt aus, dass ein Nichtrauchen Verbot wahrscheinlich in Flächen gilt, welche eine Hausnummer zugewiesen haben
und sehr unwahrscheinlich in Gebieten, die als Waldgebiete ausgeschildert sind.
Für das Angelverbot sind keine Regeln definiert. Unser Algorithmus wird sich also nur die nächstegelegene Fläche als wahrscheinlich nehmen,
oder sollten der Punkt innerhalb von 2 Flächen liegen\footnote{Es könnte eine Fläche 'Niedersachsen' und eine Fläche 'Waldgebiet' definiert sein.},
so wird die kleinere von beiden ausgewählt.

\subsection{Probleme}
Da die Berechnung der Schnittpolygone sehr aufwändig ist und wir nur eine Näherung benötigen,
wird die Berechnung der Schnittpolygone und deren Flächen vereinfach durch erstellen von Bounding Boxes.
Es wird also nicht das wirkliche Schnittpolygon berechnet, sondern die Schnitt Bounding Box, sowie die Fläche der Bounding Box.
