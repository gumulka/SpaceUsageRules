\section{Lastenheft}
Was der ``Kunde'' ausgeschrieben hat, das er haben will.\\
Aus der Aufgabenstellung des Wettbewerbs und aus den Kriterien zur Auswertung, die der Veranstalter online gestellt hat, kann man
verschiedene geforderte Bestandteile herauslesen.\\
Aus der Aufgabenstellung für die erste Runde:
\begin{itemize}
 \item bereitgestellten Testdaten herunterladen
 \item intuitives Regelwerk erstellen
 \item eine Software implementieren, die 
 \begin{itemize}
 \item einen Geltungsbereich errechnet
 \item als Eingabe SURs und Geokoordinaten hat
 \item Polygone als Ausgabe hat
 \item erweiterbar für zusätzliche SURs
 \end{itemize}
 \item die Testdaten mit der implementierten Software auswerten
 \item die Ergebnisse im beschriebenen KML-Format speichern
 \item Ergebnisse plausibilisieren
\end{itemize}
Aus der Aufgabenstellung für die zweite Runde:
\begin{itemize}
 \item selbst Datensätze erstellen
 \begin{itemize}
 \item die mindestens 10 verschiedene SURs enthalten
 \item sie in unserer Umgebung erstellen
 \item spezifiziertes Format für die IDs der SURs benutzen
 \item Fotos der zugehörigen Informationsschilder beifügen
 \item einen intuitiven Geltungsbereich angeben
 \item KML-Format und Endung \textsl{[ID].truth.kml} benutzen
 \end{itemize}
 \item eigene Datensätze mit der eigenen Software auswerten
 \item Korrektheit und Präzision der Ergebnisse diskutieren
 \begin{itemize}
 \item von den eigenen Datensätzen und vom Testdatensatz
 \item brechnete und intuitive Geltungsbereiche vergleichen
 \end{itemize}
 \item intuitiv erstellte Geltungsbereiche dürfen nicht zur Berechnung benutzt werden
\end{itemize}
Weiteres aus der Aufgabenstellung:
\begin{itemize}
 \item Dokumentationen
 \begin{itemize}
 \item Bedienungsanleitung
 \item Installationsanleitung
 \item Entscheidungen in der Software-Entwicklung und über die Auswahl von Algorithmen und Datenstrukturen dokumentieren
 \end{itemize}
\end{itemize}
Aus den Kriterien zur Auswertung:
\begin{itemize}
 \item Theoretischer Ansatz
 \begin{itemize}
 \item Recherche
 \item Modellierung
 \item Komplexitätsanalyse
 \item Aufwandsabschätzung
 \end{itemize}
 \item Dokumentation
 \begin{itemize}
 \item Konsitenz, Vollständigkeit und Nachvollziehbarkeit des Berichts
 \item Gliederung und Layout des Berichts
 \item Codedokumentation
 \end{itemize}
 \item Softwaretechnik
 \begin{itemize}
 \item klare Beschreibung der Anforderungen, Entwurfsüberlegungen und -entscheidungen
 \item klare Beschreibung der Tests, Ergebnisse und Schlussfolgerungen daraus
 \item Güte der Vorgehensweise, der Entwurfsentscheidungen und des Testkonzepts
 \end{itemize}
 \item Benutzerschnittstelle
 \begin{itemize}
 \item Verständlichkeit und Übersichtlichkeit (usability)
 \item spannende GUI-Ideen (visibility)
 \end{itemize}
 \item Funktionalität
 \begin{itemize}
 \item funktionale Korrektheit und Güte der Lösung (im Sinne der Aufgabenstellung)
 \item Nachprüfbarkeit der Testergebnisse
 \end{itemize}
\end{itemize}



