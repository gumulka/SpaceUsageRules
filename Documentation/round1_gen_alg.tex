\section{Automatisierte Regelerstellung}
Da uns die Grundaufgabe nicht informatiklastig genug ist
haben wir beschlossen unser selbst definiertes Regelwerk abschließend gegen ein Learning Classifier System antreten zu lassen.


\subsection{Eingabedaten}
\begin{itemize}
\item Eine Liste mit SpaceUsageRules wie in \fref{sec:Eingabedaten_GI} definiert.
\item Die dazugehörigen truth.kml Dateien.
\item Verschiedene Parameter zum Steuern des genetischen Algorithmus
\footnote{Es gibt hierfür Standartwerte}
  \begin{itemize}
  \item Populationsgröße
  \item Anzahl der Verbesserungsversuche
  \item Anzahl der Population, welche unverändert in die nächste Generation gehen.
  \item Anzahl der Populationen, welche mutiert in die nächste Generation gehen.
  \item Anzahl der Populationen, welche mit anderen gemischt in die nächste Generation gehen.
  \end{itemize}
\end{itemize}
\subsection{Ausgabedaten}
Das System ist so entworfen, dass es als Ausgabe Daten liefert wie in \fref{sec:Eingabedaten_Wir} verlangt werden.
\subsection{Fitnessfunktion}
Als Fitnessfunktion wird eine ähnliche Funktion wie in \fref{eq:guete} verwendet. Jedoch wird nicht $R$ insgesamt, sondern jedes $R_v$ einzeln evaluiert.
