\section{Automatisierte Regelerstellung}
Da uns die Grundaufgabe nicht informatiklastig genug ist
haben wir beschlossen unser selbst definiertes Regelwerk abschließend gegen ein Learning Classifier System antreten zu lassen. (out of date?)

\subsection{Eingabedaten}
\begin{itemize}
\item Eine Liste mit SpaceUsageRules wie in \fref{sec:Eingabedaten_GI} definiert.
\item Die dazugehörigen truth.kml Dateien.
\item Verschiedene Parameter zum Steuern des genetischen Algorithmus
%\footnote{Siehe auch fref{sec:Evaluation_genAlg}}
  \begin{itemize}
  \item Populationsgröße
  \item Anzahl der Verbesserungsversuche
  \item Anzahl der Populationen, welche unverändert in die nächste Generation gehen.
  \item Anzahl der Populationen, welche mutiert in die nächste Generation gehen.
  \item Anzahl der Populationen, welche mit anderen gemischt in die nächste Generation gehen.
  \end{itemize}
\end{itemize}

\subsection{Ausgabedaten}
Das System ist so entworfen, dass es als Ausgabe Daten liefert wie sie in \fref{sec:Eingabedaten_Wir} von unserem Algortihmus verlangt werden.

\subsection{Vorgehensweise}
Es werden zunächst für alle Eingabedaten die Daten im Umkreis der Positionsdaten aus Openstreetmap heruntergeladen.
Aus diesen werden alle möglichen Tags extrahiert. Tags, welche nur einmalig vorkommen, werden ignoriert, um zum einen die Menge zu verkleinern
und zum anderen die Regeln allgemeiner zu halten und sie so einfacher und besser auf andere Datensätze übertragen zu können.
Anschließend für jede Verbotsmenge ein Genetischer Algorithmus erstellt, welcher die Verbote für diesen optimiert und in einem eigenem Thread läuft.
Nachdem alle Algorithmen durchgelaufen sind, werden die Optimalen eingesammelt und aus ihnen ein Regelwerk erstellt und in eine Datei geschrieben.

\subsection{Fitnessfunktion}
TODO



\subsection[Evaluation]{Evaluation\protect\footnotemark}
\footnotetext{Es wurden nur die einfachen Datensätze ausgewertet.
Alle Durchläufe fanden auf dem selben Rechner statt (Core i5-3570K - 16GB RAM)}
\label{sec:Evaluation_genAlg}

\begin{figure}
  \begin{subfigure}{\textwidth}
  \includegraphics[width=\textwidth]{LaufzeitGen90Q.png}
  \caption{Qualität}
  \end{subfigure}

  \begin{subfigure}{\textwidth}
  \includegraphics[width=16cm]{LaufzeitGen90T.png}
  \caption{Laufzeit}
  \end{subfigure}
\caption{in Abhängigkeit verschiedener Paramter.}
\label{fig:GenAlgAll}
\end{figure}

Wie in \fref{fig:GenAlgAll} zu sehen ist, wird die Laufzeit und die Ergebnisse des Genetischen Algorithmus positiv beeinflusst,
wenn man den Merge-Wert möglichst hoch setzt. Kein anderer Faktor hat so einen positiven Einfluss auf das Ergebniss.
(Bei höheren Populationsgrößen steigt zwar auch die Qualität, aber gleichzeitig auch die Laufzeit im Gegensatz zum Merge-Wert.)

\begin{figure}
  \begin{subfigure}{\textwidth}
  \includegraphics[width=\textwidth]{LaufzeitMerge5Q.png}
  \caption{Qualität}
  \label{fig:QualitätMerge5}
  \end{subfigure}

  \begin{subfigure}{\textwidth}
  \includegraphics[width=\textwidth]{LaufzeitMerge5T.png}
  \caption{Laufzeit}
  \label{fig:LaufzeitMerge5}
  \end{subfigure}
\caption{des Genetischen Algorithmus für fixen Merge-Wert von 5.\protect}
\label{fig:GenAlgMerge5}
\end{figure}

Wenn wir festlegen, dass 50\% der Populationen in der neuen Generation aus den Besten der alten zusammen gepaart werden sollen,
dann erhalten wir Ergebnisse (Siehe \fref{fig:GenAlgMerge5}), die in der Qualität sehr homogen sind und für Populationsgröße
und Wiederholungen in der Laufzeit ansteigt.
Es ist also zu vermuten, dass bereits für eine kleine Anzahl von Wiederholungen und Populationsgrößen
ein brauchbares Ergebnis erzielt werden kann.
Des weiteren ist Ersichtlich, dass die Anzahl der Populationen, welche mutiert übernommen werden sollen
keinen nennenswerten Einfluss auf die Qualität oder Laufzeit hat.


Sollte das Programm später mit einem größeren Datensatz genutzt werden, schlagen wir also folgende Parameter vor:
  \begin{itemize}
\itemsep2pt
  \item Populationsgröße: 200 (500, wenn man die Rechenzeit hat.)
  \item Anzahl der Verbesserungsversuche: 100 (500, wenn man die Rechenzeit hat.)
  \item Anzahl der unveränderten Populationen: 20
  \item Anzahl der mutierten Populationen: 40
  \item Anzahl der gemergten Populationen: 100
  \end{itemize}
