\section{Automatisierte Regelerstellung}
Da uns die Grundaufgabe nicht informatiklastig genug ist
haben wir beschlossen unser selbst definiertes Regelwerk abschließend gegen ein Learning Classifier System antreten zu lassen.


\subsection{Eingabedaten}
\begin{itemize}
\item Eine Liste mit SpaceUsageRules wie in \fref{sec:Eingabedaten_GI} definiert.
\item Die dazugehörigen truth.kml Dateien.
\item Verschiedene Parameter zum Steuern des genetischen Algorithmus
%\footnote{Siehe auch fref{sec:Evaluation_genAlg}}
  \begin{itemize}
  \item Populationsgröße
  \item Anzahl der Verbesserungsversuche
  \item Anzahl der Population, welche unverändert in die nächste Generation gehen.
  \item Anzahl der Populationen, welche mutiert in die nächste Generation gehen.
  \item Anzahl der Populationen, welche mit anderen gemischt in die nächste Generation gehen.
  \end{itemize}
\end{itemize}

\subsection{Ausgabedaten}
Das System ist so entworfen, dass es als Ausgabe Daten liefert wie in \fref{sec:Eingabedaten_Wir} von unserem Algortihmus verlangt werden.

\subsection{Fitnessfunktion}

\subsection{Evaluation}
\label{sec:Evaluation_genAlg}
\begin{figure}
  \begin{subfigure}[b]{\textwidth}
  \includegraphics[width=\textwidth]{LaufzeitGen90Q.png}
  \caption{Qualität}
  \end{subfigure}

  \begin{subfigure}[b]{\textwidth}
  \includegraphics[width=16cm]{LaufzeitGen90T.png}
  \caption{Laufzeit}
  \end{subfigure}
\caption{in Abhängigkeit verschiedener Paramter.\protect\footnotemark}
\label{fig:GenAlgAll}
\end{figure}

Wie in \fref{fig:GenAlgAll} zu sehen ist, wird die Laufzeit und die Ergebnisse des Genetischen Algorithmus positiv beeinflusst,
wenn man den Merge-Wert möglichst hoch setzt. Kein anderer Faktor hat so einen hohen Einfluss auf das Ergebniss.
Auch ist ersichtlich, dass wenn man schnelle Ergebnisse haben möchte, der Algorithmus auch mit sehr kleinen
Populationen und Fehlerkorrekturwerten brauchbare Ergebnisse und kurzer Zeit liefert. (unter 5 min)

\begin{figure}
  \begin{subfigure}[b]{\textwidth}
  \includegraphics[width=\textwidth]{LaufzeitMerge5Q.png}
  \caption{Qualität}
  \label{fig:QualitätMerge5}
  \end{subfigure}

  \begin{subfigure}[b]{\textwidth}
  \includegraphics[width=\textwidth]{LaufzeitMerge5T.png}
  \caption{Laufzeit}
  \label{fig:LaufzeitMerge5}
  \end{subfigure}
\caption{des Genetischen Algorithmus für fixen Merge-Wert von 5.\protect}
\label{fig:GenAlgMerge5}
\end{figure}

Wenn wir festlegen, dass 50\% der Populationen in der neuen Generation aus den Besten der alten zusammen gepaart werden sollen,
dann erhalten wir Ergebnisse (Siehe \fref{fig:GenAlgMerge5}), die darauf schließen lassen, dass die Qualität zwar mit geringeren Wiederholungen
mehr schwankt, die allgemeine Laufzeit aber sehr gut ist.
Des weiteren ist Ersichtlich, dass die Anzahl der Populationen, bzw Anzahl der Populationen, welche mutiert übernommen werden sollen
keinen nennenswerten Einfluss auf die Qualität hat.

\footnotetext{Beim erstellen der Grafiken wurden nur die einfachen Datensätze ausgewertet.
Alle Durchläufe fanden auf dem selben Rechner statt (Core i5-3570K - 16GB RAM)}
