\section{Grundüberlegungen}
Aus den Beispielen im Text lässt sich ableiten, dass es sinnvoll ist 'nicht rauchen' Gebiete eher in Bereiche zu legen, welche Gebäude representieren,
oder alternativ Parkplätze bei der Berechnung auszublenden.

\subsection{Testdatensatz}
\label{sec:Testdatensatz}
Zunächst wurde der Testdatensatz mithilfe von Google Maps, Google Street View, Google Earth und OpenStreetMap auf Richtigkeit, Machbarkeit der Lösungen und eventuelle Fehler überprüft.
Im Testdatensatz lassen sich etliche Fehler und Unvollständigkeiten finden. So sind für die Datensätze:
\begin{itemize}
\itemsep0pt
\item 33 (nur teilweise)
\item 34 (nur teilweise)
\item 36
\item 38
\item 39
\item 41
\item 43 - 45
\item 49 (nur teilweise)
\item 50 (nur teilweise)
\item 53
\item 55 - 62
\end{itemize}
keine Representationen der beabsichtigten Geltungsbereiche in OpenStreetMap vorhanden.
\newline
\\
Weitere Fehler sind:
\begin{itemize}
\itemsep0pt
\item 16 Der Bereich müsste größer sein.
\item 33 Sollte den selben Bereich abdecken wie 24.
\item 40 Es ist kein Häuserblock gemeint, sonder nur ein einzelnes Haus auf der gegenüberliegenden Strassenseite.
\item 43 Der reale Bereich ist etwas verschoben.
\item 51 Der reale Bereich liegt auf der gegenüberliegenden Strassenseite.
\item 55 Das gewählte Gebäude ist falsch. Das reale befindet sich weiter westlich.
\item 57 Das Bild zeigt das Gebäude, welches bei Bild 41 abfotografiert wurde.
\item 59 Das Bild suggestiert, dass das Parkverbot in dem Strassenabschnitt vor dem Gebäude gilt und nicht in diesem, da es sich um ein Wohnhaus handelt.
\item 84 Das Bild suggestiert einen anderen Bereich, als auf Google Maps angezeigt wird. (in Maps ist kein See erkennbar.)
\item 85 Das Bild zeigt ein 'Betreten Verboten' Schild mit Schienen im Hintergrund und sollte dementsprechend für den Schienenbereich gelten und nicht für das Bahnhofsgebäude.
\item 90 Das Verbot sollte nur für ein einzelnen Haus gelten.
\item 93 Das Verbot gilt sinnvollerweise für den Zoo und nicht ein einzelnes Gebäude.
\item 96 Das 'Betreten Verboten im Brandfall' gilt nur für den Fahrstuhl und nicht für das ganze Gebäude.
\end{itemize}

\subsection{Fehlerbehebung}
Nach Rücksprache mit Herrn Porada, sowie Herrn Schöning haben sich einige Fehler klären lassen.
\begin{itemize}
\itemsep0pt
\item Die Fehler in 16, 33 und 59 wurden berichtigt.
\item Bei 93 wurde geklärt, dass das Schild nur für ein einzelnes Zoogebäude gelten darf/kann, da im Zoo das teilweise füttern der Tiere erlaubt ist.
\item Für 96 wurde es dabei belassen, dass man ein ganzes Haus nimmt.
\item Für 40 und 85 hat sich keine Übereinstimmung der Interpretation finden lassen.
\item Für die Restlichen wurden die Fehler als Aufnahmeunschärfe erklärt und werden nicht weiter hinterfragt sondern als richtig angenommen.
\end{itemize}


\subsection{Erweiterung der Testdaten}
Da in dem Testdatensatz so viele Fehler vorhanden sind, ist geplant diesen zu erweitern. Dazu wird eine Android-App programmiert, in welcher Nutzer SpaceUsageRules
abhängig von ihrer Position angezeigt bekommen und sie neue hinzufügen können.
Um die Qualität unserer bisherige Regeln zu überprüfen, und dem Benutzer die Eingabe zu vereinfachen, wird ein Vorschlag gemacht\footnote{Dieses wurde in der Implementierung fallen gelassen,
da die Berechnung des beabsichtigten Geltungsbereiches auf Mobilgeräten bis zu 10 Sekunden dauern kann und somit die Nutzerfreundlichkeit erheblich stört.},
welchen er annehmen oder abändern kann.
Auch soll es ihm möglich sein einen komplett neuen Bereich einzeichnen zu können.
Die angelegten Daten werden anschließend mit OpenStreetMap synchronisiert, sodass sich für die Nutzer auch ein eigener Nutzen ergibt
und sie ein Ergebnis ihres Beitrages sehen.
Für jede angelegte Regel werden Nutzungsdaten an uns übertragen.
Wir erhoffen uns daraus einen Testdatensatz zu bekommen, welcher um den Faktor 5 bis 10 größer ist und es uns erlaubt die erstellten Regeln effizienter zu überprüfen.
