\section{Regeln}

\subsection{Grundüberlegungen}
Aus den Beispielen im Text lässt sich ableiten, dass es sinnvoll ist 'nicht rauchen' Gebiete eher in Bereiche zu legen, welche Gebäude representieren,
oder alternativ Parkplätze bei der Berechnung auszublenden.

\subsection{Testdatensatz}
Zunächst wurde der Testdatensatz mithilfe von Google Maps, Google Street View, Google Earth und OpenStreetMap auf Richtigkeit, Machbarkeit der Lösungen und eventuelle Fehler überprüft.
Im Testdatensatz lassen sich etliche Fehler und Unvollständigkeiten finden. So sind für die Datensätze:
\begin{itemize}
\item 33 (nur teilweise)
\item 34 (nur teilweise)
\item 36
\item 38
\item 39
\item 41
\item 43 - 45
\item 49 (nur teilweise)
\item 50 (nur teilweise)
\item 53
\item 55 - 62
\end{itemize}
keine Representationen der beabsichtigten Geltungsbereiche in OpenStreetMap vorhanden.
\newline
\\
Weitere Fehler sind:
\begin{itemize}
\item 16 Der Bereich müsste größer sein.
\item 33 Sollte den selben Bereich abdecken wie 24.
\item 40 Es ist kein Häuserblock gemeint, sonder nur ein einzelnes Haus auf der Gegenüberliegenden Strassenseite.
\item 43 Der reale Bereich ist etwas verschoben.
\item 51 Der reale Bereich liegt auf der gegenüberliegenden Strassenseite.
\item 55 Das gewählte Gebäude ist falsch. Das reale befindet sich weiter westlich.
\item 57 Das Bild zeigt das Gebäude, welches bei Bild 41 abfotografiert wurde.
\item 59 Das Bild suggestiert, dass das Parkverbot in dem Strassenabschnitt vor dem Gebäude gilt und nicht in diesem, da es sich um ein Wohnhaus handelt.
\item 62
\item 84 Das Bild suggestert einen anderen Bereich, als auf Google Maps angezeigt wird. (in Maps ist kein See erkennbar.)
\item 85 Das Bild zeigt ein betreten verboten Schild mit Schienen im Hintergrund und sollte dementsprechend für den Schienenbereich gelten und nicht für das Bahnhofsgebäude.
\item 90 Das Verbot sollte nur für ein einzelnen Haus gelten.
\item 93 Das Verbot filt sinnvollerweise für den Zoo und nicht ein einzelnes Gebäude, welches davor liegt.
\item 96 Das betreten Verboten im Brandfall gilt nur für den Fahrstuhl und nicht für das ganze Gebäude.
\end{itemize}

\subsection{Herangehensweise}
\subsubsection{Erweiterung der Testdaten}
Da in dem Testdatensatz so viele Fehler vorhanden sind, ist geplant diesen zu erweitern. Dazu wird eine Android-App programmiert, in welcher Nutzer neue SpaceUsageRules
abhängig von ihrer Position angezeigt bekommen und sie neue hinzufügen können.
Um die Qualität unserer bisherige Regeln zu überprüfen wird dem Nutzer ein Vorschlag gemacht, welchen er annehmen oder abändern kann. Auch soll es ihm möglich sein
einen komplett neuen Bereich einzeichnen zu können.
Die angelegten Daten werden anschließend mit OpenStreetMap synchronisiert, sodass sich für die Nutzer auch ein eigener Nutzen ergibt und sie ein Ergebnis ihres Beitrages sehen.
Für jede angelegte Regel werden Nutzungsdaten an uns übertragen, die aus der Position des Nutzers, des Regelnahmens sowie des angelegten Bereiches besteht.
Wir erhoffen uns daraus einen Testdatensatz zu bekommen, welcher um den Faktor 5 bis 10 größer ist und es uns erlaubt die erstellten Regeln effizienter zu überprüfen.

\subsubsection{Regelbeschreibung}
Da in fast allen Fällen ein sehr nah gelegenes Polygon gemeint ist, wird der Abstand aller Vorhandenen Polygone nach unseren Regeln gewichtet
um anschließend das Polygon auszuwählen, welches nach der Gewichtung den kürzesten Abstand hat.


\subsection{Formale Beschreibung:}
Sei $ID$ die die Bezeichnung eines Bildes.\\
Sei $t$ ein $tag (key \to value)$ aus OpenStreetMap.\\
Sei $r(t)$ eine Regel $t \to [0..2]$ \\
Sei $Z_{X}$ die Menge der Tag $t$ für die Gilt: $t$ kommt in X vor. \\
Sei $R_v$ eine Menge von Funktionen $r$ für das Verbot $v$\\
Sei $C_{ID}$ die Geocoordinate, an welcher das Bild zu $ID$ aufgenommen wurde. \\
Sei $D_{X,ID}$ der Abstand eines Polygons $X$ zu $C_{ID}$ \\
Sei $O_{ID}$ eine Menge von Polygonen aus OpenStreetMap in der Umgebung zu $C_{ID}$\\
Sei $A_{X}$ die Fläche des Polygons $X$\\
Sei $V$ die Menge aller Verbote $v$.\\
Sei $V_{ID}$ die Menge der Verbote für die gilt: $v$ wird auf dem Bild zu $ID$ abgebildet.\\
Sei $S_{v}$ die Menge der $ID$ für die gilt: $v \in V_{ID}$\\
Sei $P_{ID}$ das gegebene Lösungspolygon zu $ID$.\\
Sei $I(X,Y)$ das Schnittpolygon der Polygone X und Y\\
\\
Sollte es für ein $t$ kein $r(t)$ geben, so wird die standartregel:
$r(t) = 1$ angewendet.

\begin{equation}
Gew(R,ID,X) := \sum_{v \in V_{ID}} \Big((1 + D_{X,ID})\prod_{t \in Z_X} \prod_{r \in R_v} r(t)\Big)
\end{equation}

\begin{align}
G(ID,R) := \{X \in O_{ID} | & Gew(R,ID,X) < Gew(R,ID,Y)\\
& \lor \Big(Gew(R,ID,X)=Gew(R,ID,Y) \land A_X < A_Y\Big) \forall Y\} \notag
\end{align}

\subsection{Beispiel}
Eine Regel kann z.B. so aussehen:
Für das Verbot Nichtrauchen: wenn ein Tag den Key 'building' enhält dann halbiere den Abstand. \\
Oder Formal: \\
$r('building') = 0.5;$\\
Sei dies unsere einzige Regel in $R_{nichtrauchen}$
\\
\newline
Seien in der Umgebung zu $ID$ nur zwei Flächen A,B in OpenStreetMap vorhanden.
\begin{itemize}
\item A enthält die Tags: 'building -> yes', 'addr:housenumber -> 34' und hat den Abstand 0.002;
\item B enthält die Tags: 'landuse->residential' und hat den Abstand 0;
\end{itemize}
Daraus errechnet sich:
\begin{itemize}
\item $Gew(R,ID,A) = (1+0.002) * 0.5 * 1 = 0.501$
\item $Gew(R,ID,B) = (1+0.0) * 1 = 1$
\end{itemize}

$G(ID,R) = A$

\subsection{Überprüfung der Regelsätze}
Um die güte der Verschiedenen Regeln zu überprüfen wird ein Programm geschrieben, welche für jede Regel die die Schnittpolygone von unseren Polygonen mit
den Musterlösungen bildet und anschließen durch je die Fläche unseres Polygons, sowie des Musterlösungspolygons teilt.
Von beiden Zahlen wird das Minimun genommen und als Indicator der Lösungsgüte angesehen.\\
\newline
\\
\begin{equation}
f(v,R) := \sum_{q\in S_v} min(\frac{I(P_q,G(q,R))}{P_q},\frac{I(P_q,G(q,R))}{G(q,R)})
\end{equation}
Um die Güte des Regelsatzes für ein einzelnes Verbot vergleichbar zu anderen Verboten zu machen wird abschließend der Indicator durch
die Anzahl der $ID$ welches es beeinflussen geteilt.\\
\begin{equation}
Güte(R) := \sum_{v \in V} \frac{ f(v,R)}{\#S_v}
\end{equation}

Wir suchen also einen Regelsatz $R$ für den $Güte(R)$ maximal ist.

\subsection{Mögliche Erweiterung}
Um den Fall abzudecken, in welchem in OpenStreetMap keine ausreichenden Daten vorhanden sind, wird um den Punkt $C_{ID}$ ein
Achteck\footnote{Relativ geringer Rechenaufwand bei gleichzeitig bester Annäherung an einen Kreis.} mit Radius $b$ gelegt
und dieses mit der gewichtet nächsten Großfläche geschnitten.
Eine Großfläche X definiert sich dadurch, dass ihre Fläche $A_X$ einen schwellwert $u$ überschreitet.
Es ist zu prüfen ob es sinnvoll ist dieses generell zu tun oder für bestimmte Regel ausser kraft zu setzen wie
z.B. ein Verbot nach offenem Feuer, welches in einem ganzen Parkgebiet gelten wird.
