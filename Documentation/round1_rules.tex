\section{Regeln}

\subsection{Formale Beschreibung:}
Sei $ID$ die die Bezeichnung eines Bildes.\\
Sei $t$ ein Tag $(key \to value)$ aus OpenStreetMap.\\
Sei $r(t)$ eine Regel $t \to [0..2]$ \\
Sei $Z_{X}$ die Menge der Tag $t$ für die Gilt: $t$ kommt in X vor. \\
Sei $R_v$ eine Menge von Funktionen $r$ für das Verbot $v$\\
Sei $C_{ID}$ die Geocoordinate, an welcher das Bild zu $ID$ aufgenommen wurde. \\
Sei $D_{X,ID}$ der Abstand eines Polygons $X$ zu $C_{ID}$ \\
Sei $O_{ID}$ eine Menge von Polygonen aus OpenStreetMap in der Umgebung zu $C_{ID}$\\
Sei $A_{X}$ die Fläche des Polygons $X$\\
Sei $V$ die Menge aller Verbote $v$.\\
Sei $V_{ID}$ die Menge der Verbote für die gilt: $v$ wird auf dem Bild zu $ID$ abgebildet.\\
Sei $S_{v}$ die Menge der $ID$ für die gilt: $v \in V_{ID}$\\
Sei $P_{ID}$ das gegebene Lösungspolygon zu $ID$.\\
Sei $I(X,Y)$ das Schnittpolygon der Polygone X und Y\\
\\
Sollte es für ein $t$ kein $r(t)$ geben, so wird die standartregel:
$r(t) = 1$ angewendet.

\begin{equation}
Gew(R,ID,X) := \sum_{v \in V_{ID}} \Big((1 + D_{X,ID})\prod_{t \in Z_X} \prod_{r \in R_v} r(t)\Big)
\end{equation}

\begin{align}
G(ID,R) := \{X \in O_{ID} | & Gew(R,ID,X) < Gew(R,ID,Y)\\
& \lor \Big(Gew(R,ID,X)=Gew(R,ID,Y) \land A_X < A_Y\Big) \forall Y\} \notag
\end{align}

\subsection{Beispiel}
Eine Regel kann z.B. so aussehen:
Für das Verbot Nichtrauchen: wenn ein Tag den Key 'building' enhält dann halbiere den Abstand. \\
$R_{nichtrauchen} = [r('building') = 0.5]$\\
\newline
Seien in der Umgebung zu $ID$ nur zwei Flächen A,B in OpenStreetMap vorhanden.
\begin{itemize}
\item A enthält die Tags: 'building -> yes', 'addr:housenumber -> 34' und hat den Abstand 0.002;
\item B enthält die Tags: 'landuse->residential' und hat den Abstand 0;
\end{itemize}
Daraus errechnet sich:
\begin{itemize}
\item $Gew(R,ID,A) = (1+0.002) * 0.5 * 1 = 0.501$
\item $Gew(R,ID,B) = (1+0.0) * 1 = 1$
\end{itemize}

$G(ID,R) = A$

\subsection{Überprüfung der Regelsätze}
Um die Güte der verschiedenen Regeln zu überprüfen wird ein Programm geschrieben, welche für jede Regel die die Schnittpolygone von unseren Polygonen mit
den Musterlösungen bildet und anschließen durch je die Fläche unseres Polygons, sowie des Musterlösungspolygons teilt.
Von beiden Zahlen wird das Minimun genommen und als Indicator der Lösungsgüte angesehen.\\
\begin{equation}
f(v,R) := \sum_{q\in S_v} min(\frac{I(P_q,G(q,R))}{P_q},\frac{I(P_q,G(q,R))}{G(q,R)})
\end{equation}
Um die Güte des Regelsatzes für ein einzelnes Verbot vergleichbar zu anderen Verboten zu machen wird abschließend der Indicator durch
die Anzahl der $ID$s welches es beeinflussen geteilt.\\
\begin{equation}
\label{eq:guete}
Güte(R) := \sum_{v \in V} \frac{ f(v,R)}{\#S_v}
\end{equation}

Wir suchen also einen Regelsatz $R$ für den $Güte(R)$ maximal ist.

\subsection{Mögliche Erweiterung}
Um den Fall abzudecken, in welchem in OpenStreetMap keine ausreichenden Daten vorhanden sind, wird um den Punkt $C_{ID}$ ein
Achteck\footnote{Relativ geringer Rechenaufwand bei gleichzeitig bester Annäherung an einen Kreis.} mit Radius $b$ gelegt
und dieses mit der gewichtet nächsten Großfläche geschnitten.
Eine Großfläche X definiert sich dadurch, dass ihre Fläche $A_X$ einen schwellwert $u$ überschreitet.
Es ist zu prüfen ob es sinnvoll ist dieses generell zu tun oder für bestimmte Regel ausser kraft zu setzen wie
z.B. ein Verbot nach offenem Feuer, welches in einem ganzen Parkgebiet gelten wird.
