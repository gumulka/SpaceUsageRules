\section{Softwareimplementierung zur Testdatensatzerweiterung}
\label{sec:HelferApp}
Um die Testdatensätze zu erweitern wird eine Android App geschrieben, welche es dem
Benutzer erlaubt weitere SpaceUsageRules anhand seines aktuellen Standorts einzugeben.

\subsection{Nutzerbasis}
Am Anfang wurde noch geplant, die App nur an Freunde zu verteilen und auf deren Mitarbeit zu hoffen.
Da wir in Freundeskreis einige Sonder- und Sozialpädagogen haben wurde überlegt für diese einen Mehrwert zu bieten,
indem wir die Möglichkeit geben, die App in den Einstellungen umzukonfigurieren, dass nur Daten zu Barrierefreiheit angezeigt und bearbeitet werden können.

Dies sorgt dafür, dass einige Klicks weniger notwendig sind und auch weniger Daten von den OSM-Servern heruntergeladen werden müssen
und sorgt hoffentlich dafür, dass die Nutzer einen Sinn für sich in der App sehen und diese nicht nur aus Freundlichkeit benutzen.

In späteren Gesprächen mit anderen potentiellen Nutzern stellte sich heraus, dass diese bereit wären die App zu nutzen,
wenn sie ihnen nur Daten zu Rauchverboten anzeigen würden.
Also wurde sie noch einmal verändert, sodass sie einem in den Einstellungen erlaubt die Suche und hinzufügen auf einen beliebigen Tag zu beschränken.

\subsection{Standortbestimmung}
Da Android Geräte generell eine Möglichkeit bieten zur Standortbestimmung, z.B. Anhand von GPS oder Netzwerkdaten,
wird diese ausgenutzt und als Startwert benutzt.
In einer späteren Phase des Projekts wurde dem Nutzer die Möglichkeit gegeben, sich Informationen zu einem beliebigen Punkt auf der Karte geben zu lassen,
indem er einfach länger auf die Karte drückt. In diesem Fall wird auch der Standort auf die gewählte Kartenposition gesetzt.

\subsection{OpenStreetMap}
Es werden Daten von OSM genutzt um den Nutzer bereits bekannte Nutzungsregeln anzuzeigen.
Damit dieser bei Änderungen keinen eigenen Account anlegen muss, wird von uns ein Account für die Nutzung mit der App angelegt,
in welchem alle Änderungen, welche in Zusammenhang mit der App stehen gemacht werden.
In einer späteren Version der App kann das evtl. geändert werden, aber aktuell erscheint es uns als die beste Möglichkeit
um nicht zu viele Leute vom hinzufügen dadurch abzuschrecken, dass sie sich irgendwo anmelden müssen,
oder aber ihre Nutzerdaten an uns weiter geben zu müssen um etwas hinzuzufügen.
Nach Beendigung des Wettbewerbs soll die App auf OAuth umgestellt werden und den Nutzern auch weiterhin zur Verfügung stehen.

Um Änderungen der Regeln zu speichern wird die API Version 0.6 benutzt und für jede Änderung ein Changeset erstellt und geschlossen.
Dabei wird darauf geachtet, dass ausser einer vorgegebenen Menge von Tags keine Daten geändert werden.

\subsection{Unser Nutzen}
Wenn der Nutzer neue Daten zu OSM hinzufügt, dann wird nicht nur der Datensatz an OSM gesendet, sondern auch an uns, zusammen mit der Position des Nutzers.

Um genau zu sein, bekommen wir folgende Daten:
\begin{itemize}
\item Die aktuelle Position des Nutzers
\item Die OSM-ID des/der ausgewählten Polygons/Polygone
\item \sout{Eine Information, ob das Polygon neu erstellt wurde}\footnote{Erkennt man daran, dass die Version des Polygons 1 ist.}
\item Sämtliche Nutzungsregeln, welche benutzt wurden
\end{itemize}
Sollte der Wettbewerb vorbei sein, oder wir genug Daten gesammelt haben wird die
Datenübertragung an uns durch ein Update der App beendet.

\subsection{Hinderlichkeiten}
Durch diese Vorgehensweise haben wir leider keine Bilder, welche Helfen würden die Angaben der Benutzer zu verifizieren,
also müssen wir davon ausgehen, dass alles richtig ist.

\subsection{Veröffentlichung}
Mitlerweile ist die App fertig und unter der URL \url{https://play.google.com/store/apps/details?id=de.uni_hannover.inma} im Google Play Store veröffentlicht.
