Man braucht folgende Daten: Data.txt, rules.xml. Außerdem ist eine Internetverbindung zu openstreetmap.org nötig.\\
Eine Manpage lässt sich mit \verb|man -l inma.1| aufrufen.
\section{Eingabedaten}
Die Data.txt enthält die wichtigsten Eingabedaten. Sie ist genauso aufgebaut wie die im Testdatensatz zum Wettbewerb bereitgestellte Datei.
In der ersten Zeile steht die Anzahl der folgenden Zeilen. Jede Zeile enthält am Ende eine Flächennutzungsregel. Am Beginn der Zeile steht
die Nummer des Testdatensatzes, zu dem die Regel gehört. Gefolgt wird diese Nummer von einem Koordinatenpaar, das den Standort der
entsprechneden Verbotsschilder beschreibt.\\
Die Rules.xml enthält die Bewertungsregeln, die das Programm benötigt. (Für deren Aufbau siehe unten.)
\section{Ausführung}
Die Standardausführung sieht folgendermaßen aus:\\
\verb|java -jar InMa.jar --data Data.txt --rules Rules.xml|
\section{Optionen}
\manoption{-h, --help}{Shows the help text.}
\manoption{-d, --data FILE}{Specifies  the  file  containing  the space usage rules and their locations.}
\manoption{-r, --rules FILE}{Specifies the file  containing  rules  for  evaluating  the environment of the location.}
\manoption{-t, --threads NUMBER}{Specifies the number of threads.}
\manoption{-u, --overlap FILE}{Specifies  the  file containing overlap values, that should be reached to be successfull.}
\manoption{-o, --outputDir PATH}{Specifies the folder where output files will be stored. Can be overwritten for the images by -i.}
\manoption{-p, --path PATH}{Specifies   the  folder  where  the  following  files  are: Data.txt, Rules.xml and optional Overlap.txt.  -d,  -r  and -u are not needed and can overwrite -p.}
\manoption{-i, --image [PATH]}{Images are created to visualise the processed data. An output directory is optional.}
\manoption{-l, --height NUMBER}{Specifies the height of the images.}
\manoption{-w, --width NUMBER}{Specifies the width of the images.}
\section{Ausgabe}
Als Ausgabe erhält man kml-Dateien, die die errechneten Gebiete enthalten. Sie können mit Google-Earth betrachtet werden.
