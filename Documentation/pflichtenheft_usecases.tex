%\subsection{Use Case-Diagramm}
%TODO Diagramm, wie die verschiedenen Use Cases, Nutzer und Systeme zusammen arbeiten.\\
\subsection{Use Case-Beschreibungen}
% Diese Nummerierung, weil es im Prinzip immer der gleich Anwendungsfall ist, mit leichten Abänderungen.
1: Einfaches Benutzen der Software in der Komandozeile, d.h. mit den Parametern --data, --rules und --outputDir.\\
1.1: Zusätzlich mit dem Parameter --threads\\
1.2: Nur mit dem Parameter --path\\
2: Einfaches Benutzen mit Fehlervisualisierung.
2.1: Zusätzlich mit dem Parameter --overlap
3: Nur mit dem Parameter --help\\


\begin{tabular}{| l | l |}
 \hline
 \textbf{Use Case Nr. 1} & Standardausführung\\
 \hline
 Umfeld & Komandozeile\\
 \hline
 Systemgrenzen & Komandozeile\\
 \hline
 Ebene & komplette Programmbenutzung\\
 \hline
 Hauptakteur & Benutzer\\
 \hline
 Stakeholder u. Interesse & Benutzer will einen Datensatz berechnen lassen.\\
 \hline
 Voraussetzung & Internetverbindung zu OSM. Angegebene Verzeichnisse existieren.\\
	      & Data.txt und Rules.xml befinden sich an angegebener Stelle. \\
 \hline
 Garantien & Sind OSM-Daten unvollständig, wird ein kreisförmiger Geltungsbereich angegeben.\\
	  & Sind SURs unbekannt, gibt es dennoch eine Ausgabe.\\
 \hline
 Erfolgsfall & Es wird eine kml-Datei mit dem berechneten Lösungspolygon im\\
	    & angegebenen Verzeichnis erstellt.\\
 \hline
 Auslöser & Der Befehl wird abgeschickt.\\
 \hline
 Beschreibung & Ein normaler Programmablauf. Folgender Befehl wird abgeschickt.\\
	    & \verb|java -jar InMa.jar --data <file> --rules <file>|\\
	    & \hspace{24pt} \verb|--ouputDir <path>|\\
	    & Daten werden von OSM herunter geladen.\\
	    & Regeln werden angewendet.\\
	    & Lösungen werden im KML-Format gespeichert.\\
 \hline
 Erweiterungen & Mit zusätzlichen oder anderen Parametern. Siehe Use Cases 1.1 bis 1.6\\
 \hline
 Technologie & -\\
 \hline
\end{tabular}



\begin{tabular}{| l | l |}
 \hline
 \textbf{Use Case Nr. 1.1} & Multithreading\\
 \hline
 Umfeld & Komandozeile\\
 \hline
 Systemgrenzen & Komandozeile\\
 \hline
 Ebene & komplette Programmbenutzung\\
 \hline
 Hauptakteur & Benutzer\\
 \hline
 Stakeholder u. Interesse & Benutzer will einen Datensatz berechnen lassen und Zeit sparen,\\
			  & indem die Arbeit auf mehrere Threads verteilt wird.\\
 \hline
 Voraussetzung & Internetverbindung zu OSM. Angegebene Verzeichnisse existieren.\\
	      & Data.txt und Rules.xml befinden sich an angegebener Stelle. \\
 \hline
 Garantien & Sind OSM-Daten unvollständig, wird ein kreisförmiger Geltungsbereich angegeben.\\
	  & Sind SURs unbekannt, gibt es dennoch eine Ausgabe.\\
 \hline
 Erfolgsfall & Es wird eine kml-Datei mit dem berechneten Lösungspolygon im\\
	    & angegebenen Verzeichnis erstellt. Alles in kürzerer Zeit.\\
 \hline
 Auslöser & Abschicken des Befehls\\
 \hline
 Beschreibung & \verb|java -jar InMa.jar --data <file> --rules <file>|\\
	      & \hspace{24pt}\verb|--ouputDir <path> --threads <int>|\\
	      & Threads werden erstellt und arbeiten.\\
 \hline
 Erweiterungen & -\\
 \hline
 Technologie & -\\
 \hline
\end{tabular}


\begin{tabular}{| l | l |}
 \hline
 \textbf{Use Case Nr. 1.2} & verkürzte Standardausführung\\
 \hline
 Umfeld & Komandozeile\\
 \hline
 Systemgrenzen & Komandozeile\\
 \hline
 Ebene & komplette Programmbenutzung\\
 \hline
 Hauptakteur & Benutzer\\
 \hline
 Stakeholder u. Interesse & Benutzer will einen Datensatz berechnen lassen.\\
 \hline
 Voraussetzung & Internetverbindung zu OSM. Angegebene Verzeichnisse existieren.\\
        & Data.txt und Rules.xml befinden sich an angegebener Stelle. \\
 \hline
 Garantien & Sind OSM-Daten unvollständig, wird ein kreisförmiger Geltungsbereich angegeben.\\
    & Sind SURs unbekannt, gibt es dennoch eine Ausgabe.\\
 \hline
 Erfolgsfall & Es wird eine kml-Datei mit dem berechneten Lösungspolygon im\\
      & angegebenen Verzeichnis erstellt.\\
 \hline
 Auslöser & Der Befehl wird abgeschickt.\\
 \hline
 Beschreibung & Folgender Befehl wird abgeschickt.\\
      & \verb|java -jar InMa.jar --path <path>|\\
      & Daten werden von OSM herunter geladen.\\
      & Regeln werden angewendet.\\
      & Lösungen werden im KML-Format gespeichert.\\
 \hline
 Erweiterungen & -\\
 \hline
 Technologie & -\\
 \hline
\end{tabular}



\begin{tabular}{| l | l |}
 \hline
 \textbf{Use Case Nr. 2} & Regelüberprüfung \\
 \hline
 Umfeld & Komandozeile\\
 \hline
 Systemgrenzen & Komandozeile\\
 \hline
 Ebene & komplette Programmbenutzung\\
 \hline
 Hauptakteur & Benutzer\\
 \hline
 Stakeholder u. Interesse & Benutzer will die bisherigen Regeln anhand von Grundwahrheiten überprüfen.\\
 \hline
 Voraussetzung & Internetverbindung zu OSM. Angegebene Verzeichnisse existieren.\\
	      & Data.txt und Rules.xml befinden sich an angegebener Stelle. \\
	      & Es befinden sich $ID$.truth.kml Dateien im gleichen Verzeichnig wie Data.txt \\
 \hline
 Garantien & Sind OSM-Daten unvollständig, wird ein kreisförmiger Geltungsbereich angegeben.\\
	  & Sind SURs unbekannt, gibt es dennoch eine Ausgabe.\\
 \hline
 Erfolgsfall & Es wird eine kml-Datei mit dem berechneten Lösungspolygon im\\
	    & angegebenen Verzeichnis erstellt.\\
	    & Für jedes Polygon, keine Überlappung von mindestens 95\% mit dem \\
	    & Lösungspolygon aufweist werden 2 Bilder erstellt, welche den Fehler visualisieren.\\
 \hline
 Auslöser & Der Befehl wird abgeschickt.\\
 \hline
 Beschreibung & Folgender Befehl wird abgeschickt.\\
	    & \verb|java -jar InMa.jar --data <file> --rules <file>|\\
	    & \hspace{24pt} \verb|--ouputDir <path>|\\
	    & Daten werden von OSM herunter geladen.\\
	    & Regeln werden angewendet.\\
	    & Bilder zu den unzureichenden Lösungen werden gespeichert.\\
	    & Lösungen werden im KML-Format gespeichert.\\
 \hline
 Erweiterungen & siehe 2.1\\
 \hline
 Technologie & -\\
 \hline
\end{tabular}


\begin{tabular}{| l | l |}
 \hline
 \textbf{Use Case Nr. 2.1} & Regelüberprüfung mit varierender Genauigkeit.\\
 \hline
 Umfeld & Komandozeile\\
 \hline
 Systemgrenzen & Komandozeile\\
 \hline
 Ebene & komplette Programmbenutzung\\
 \hline
 Hauptakteur & Benutzer\\
 \hline
 Stakeholder u. Interesse & Benutzer will die bisherigen Regeln anhand von Grundwahrheiten überprüfen.\\
 \hline
 Voraussetzung & Internetverbindung zu OSM. Angegebene Verzeichnisse existieren.\\
        & Data.txt, Rules.xml und overlap.txt befinden sich an angegebener Stelle. \\
        & Es befinden sich $ID$.truth.kml Dateien im gleichen Verzeichnig wie Data.txt \\
 \hline
 Garantien & Sind OSM-Daten unvollständig, wird ein kreisförmiger Geltungsbereich angegeben.\\
    & Sind SURs unbekannt, gibt es dennoch eine Ausgabe.\\
 \hline
 Erfolgsfall & Es wird eine kml-Datei mit dem berechneten Lösungspolygon im\\
      & angegebenen Verzeichnis erstellt.\\
      & Für jedes Polygon, keine Überlappung von mindestens 95\% oder in overlap.txt \\
      & angegebenem Wert mit dem Lösungspolygon aufweist werden 2 Bilder erstellt,\\
      & welche den Fehler visualisieren.\\
 \hline
 Auslöser & Der Befehl wird abgeschickt.\\
 \hline
 Beschreibung & Folgender Befehl wird abgeschickt.\\
      & \verb|java -jar InMa.jar --data <file> --rules <file>|\\
      & \hspace{24pt} \verb|--ouputDir <path>|\\
      & Daten werden von OSM herunter geladen.\\
      & Regeln werden angewendet.\\
      & Bilder zu den unzureichenden Lösungen werden gespeichert.\\
      & Lösungen werden im KML-Format gespeichert.\\
 \hline
 Erweiterungen & -\\
 \hline
 Technologie & -\\
 \hline
\end{tabular}


\begin{tabular}{| l | l |}
 \hline
 \textbf{Use Case Nr. 3} & Hilfetext\\
 \hline
 Umfeld & Kommandozeile\\
 \hline
 Systemgrenzen & Kommandozeile\\
 \hline
 Ebene & komplette Programmausführung\\
 \hline
 Hauptakteur & Benutzer\\
 \hline
 Stakeholder u. Interesse & Der Benutzer möchte sich den Hilfetext ausgeben lassen.\\
 \hline
 Voraussetzung & -\\
 \hline
 Garantien & Der Hilfetext wird angezeigt.\\
 \hline
 Erfolgsfall & Der Hilfetext wird angezeigt.\\
 \hline
 Auslöser & Abschicken des Befehls\\
 \hline
 Beschreibung & Folgender Befehl wird abgeschickt.\\
	      & \verb|java -jar InMa.jar --help|\\
	      & Der Hilfetext wird angezeigt.\\
 \hline
 Erweiterungen & -\\
 \hline
 Technologie & -\\
 \hline
\end{tabular}


% Beispieltabelle:\\
% \begin{tabular}{| l | l |}
%  \hline
%  \textbf{Use Case Nr. 1} & Name des Use Cases\\
%  \hline
%  Umfeld & \\
%  \hline
%  Systemgrenzen & \\
%  \hline
%  Ebene & \\
%  \hline
%  Hauptakteur & \\
%  \hline
%  Stakeholder u. Interesse & \\
%  \hline
%  Voraussetzung & \\
%  \hline
%  Garantien & \\
%  \hline
%  Erfolgsfall & \\
%  \hline
%  Auslöser & \\
%  \hline
%  Beschreibung & \\
%  \hline
%  Erweiterungen & \\
%  \hline
%  Technologie & \\
%  \hline
% \end{tabular}
