\documentclass{beamer}

\usepackage[utf8]{inputenc}
\usepackage{default}

\usetheme{Warsaw}
\usecolortheme{wolverine}
% Vorschläge für alternative themes
% Szeged-dolphin
% Rochester-sidebartab
% Montpellier-whale
% Ilmenau-orchid
% etwas eigenes

\title[Wettbewerbsbeitrag] % (optional, only for long titles)
{Wettbewerbsbeitrag zum InformatiCup}
\subtitle{Team InMa}
\author[Pflug, Zilz] % (optional, for multiple authors)
{F.~Pflug\inst{1} \and P.~Zilz\inst{2}}
\institute[Leibniz Universität Hannover] % (optional)
{
  \inst{1}% oder ein besttimmtes Institut
  Fakultät für Elektrotechnik und Informatik\\
  Leibniz Universität Hannover
  \and
  \inst{2}%
  Fakultät für Mathematik und Physik\\
  Leibniz Universität Hannover
}
\date[Informatiktage 2015] % (optional)
{Informatiktage, 2015}
\subject{Informatik}


\begin{document}

\frame{\titlepage}

\begin{frame}
    \frametitle{Datensatz untersuchen}
    %Content goes here
\end{frame}

\begin{frame}
    \frametitle{Fehlerbereinigung}
    %Content goes here
\end{frame}

\begin{frame}
    \frametitle{Idee zur Android-App}
    %Content goes here
\end{frame}

\begin{frame}
    \frametitle{Darstellung des Datensatzes}
    %Content goes here
\end{frame}

\begin{frame}
    \frametitle{banaler Algorithmus}
    %Content goes here
\end{frame}

\begin{frame}
    \frametitle{Entfernung gewichten durch Tags}
    %Content goes here
\end{frame}

\begin{frame}
    \frametitle{Umgebungspolygon schneiden}
    %Content goes here
\end{frame}

\begin{frame}
    \frametitle{rules.xml Format}
    %Content goes here
\end{frame}

\begin{frame}
    \frametitle{Training mit vorhandenen Daten}
    %Content goes here
\end{frame}

\begin{frame}
    \frametitle{App um mehr Daten zu bekommen}
    %Content goes here
\end{frame}


\end{document}
